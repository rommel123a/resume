% !TEX TS-program = xelatex
% !TEX encoding = UTF-8 Unicode
% !Mode:: "TeX:UTF-8"

\documentclass{resume}
\usepackage{zh_CN-Adobefonts_external} % Simplified Chinese Support using external fonts (./fonts/zh_CN-Adobe/)
% \usepackage{NotoSansSC_external}
% \usepackage{NotoSerifCJKsc_external}
% \usepackage{zh_CN-Adobefonts_internal} % Simplified Chinese Support using system fonts
\usepackage{linespacing_fix} % disable extra space before next section
\usepackage{cite}

\begin{document}
\pagenumbering{gobble} % suppress displaying page number

\name{王哲凯}

\basicInfo{
  \email{505502416@qq.com} \textperiodcentered\
  \phone{(+86) 158-0729-6613} \textperiodcentered\ 

\section{\faGraduationCap\ 教育背景}
\datedsubsection{\textbf{华中科技大学(HUST)}, 武汉, 湖北}{2018 -- Present}
\textit{本科在读} ,计算机科学与技术,预计2022年6月毕业
\\
\section{\faUsers\ 项目经历}
\datedsubsection{\textbf{Unreal} ,第三人称射击游戏}{2020.8 -- 2020.8}
项目《HUSTer:GO》是一款基于Unreal进行开发的第三人称射击游戏
\begin{itemize}
  \item 服务器同步
  \item 动画状态机实现和人物控制代码实现
  \item Character,游戏模式,游戏状态,GameInstance等改进
  \item UI蓝图及逻辑实现
\end{itemize}
\\
% Reference Test
%\datedsubsection{\textbf{Paper Title\cite{zaharia2012resilient}}}{May. 2015}
%An xxx optimized for xxx\cite{verma2015large}
%\begin{itemize}
%  \item main contribution
%\end{itemize}

\section{\faCogs\ IT技能}
\begin{itemize}[parsep=0.5ex]
  \item 编程语言: C++ > C > C\#
  \item 平台: Windows > Linux
  \item 框架: 接触过Qt-C++, 了解Unreal
  \item 熟练掌握基本常用数据结构的构造与其在实际工程中的应用
  \item 熟练掌握计算机网络基础知识,掌握Socket编程基础知识
  \item 掌握git的基本操作, 了解Perforce的基本操作
  \item 掌握C++11的部分新特性:move语义,functional和lambda, 了解unreal-C++关于委托特性的实现
  \item 了解Unreal开发代码规范和资源命名管理
  \item 了解SQLite-C++接口函数和基本语法
  \item 了解lua的基本语法
\end{itemize}
\\
\section{\faHeartO\ 获奖经历}
\datedline{\textit{求是杯(挑战杯本校校赛)}, 铜奖 }{2020.10}
\datedline{\textit{湖北省十佳社会实践队伍}, 队长 }{2019.10}
\\
\section{\faInfo\ 其他}
\begin{itemize}[parsep=0.5ex]
  \item 外语水平: CET-4:606
  \item GitHub: https://github.com/rommel123a
  \item 沟通能力良好,组织能力强,团队协作能力强
\end{itemize}

%% Reference
%\newpage
%\bibliographystyle{IEEETran}
%\bibliography{mycite}
\end{document}
